\documentclass[../../thesis.tex]{subfiles}

\begin{document}

\chapter{Neural Network Loss Surfaces, Critical Points, and Critical Point-Finding Algorithms}\chapterlabel{one}

\begin{figure}\centering
\parbox{.4\textwidth}{\centering
\begin{picture}(70,70)
\put(0,50){\framebox(20,20){}}
\put(10,60){\circle*{7}}
\put(50,50){\framebox(20,20){}}
\put(60,60){\circle*{7}}
\put(20,10){\line(1,0){30}}
\put(20,10){\line(-1,1){10}}
\put(50,10){\line(1,1){10}}
\end{picture}
\caption{The author before embarking on the PhD.}}
\hfill
\parbox{.4\textwidth}{\centering
\begin{picture}(70,70)
\put(0,50){\framebox(20,20){}}
\put(10,60){\circle*{7}}
\put(50,50){\framebox(20,20){}}
\put(60,60){\circle*{7}}
\put(20,10){\line(1,0){30}}
\put(20,10){\line(-1,-1){10}}
\put(50,10){\line(1,-1){10}}
\end{picture}
\caption{The author on completing the PhD.}}
\end{figure}

This is \chapterref{one}.

\section{Intro}\sectionlabel{intro}

Neural networks are cool but mysterious~\cite{lecun2015}.

\section{Notation}\sectionlabel{notation}

Forget \sectionref{intro}.
We need to go over notation,
from $\cA$ to $\cZ$.

A vector is $\vec{x}$.
If it's a random variable vector,
$\vec{x}\sim\Normal\left(0, \Sigma)$,
we might write $\Ex{x} = 0$ and
$\Var\left(x\right) = \Sigma$.

An estimate of $\theta$ is $\widebar{\theta}$,
or perhaps $\hat{\theta}$.

We sometimes punctuate our formulas,
like in \equationref{eg} below:
\begin{equation}\equationlabel{eg}
	\hat{\mu} \defeq \frac{1}{n} \sum_{x_i \in X} x_i\mper
\end{equation}

We'll often worry about the norm of a vector,
$\normt{\vec{x}}$.
Recall that the norm, which is a map
$\normt{\cdot} \from \R^n \to \R$,
is defined via
\[
	\snormt{\vec{x}} \defeq \trspvec{x}\vec{x}
\]
When unambiguous, we will simply write $\norm{\vec{x}}$
and $\snorm{\vec{x}}$.

The linear subspace a matrix $M$ maps to $\vec{0}$
is denoted by $\ker{M}$,
and its orthogonal complement by $\co{\ker{M}}$.

This will be important whenever
$\grad{f}{\theta} \in \ker\hess{f}{\theta}$,
and
$\sgn{f}{\theta} > 0$.

\onlyinsubfile{\printbibliography}

\end{document}
