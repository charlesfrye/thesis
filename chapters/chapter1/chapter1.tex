\documentclass[../../thesis.tex]{subfiles}

\begin{document}

\chapter{Neural Network Loss Surfaces, Critical Points, and Critical Point-Finding Algorithms}\chapterlabel{one}

\begin{figure}\centering
\parbox{.4\textwidth}{\centering
\begin{picture}(70,70)
\put(0,50){\framebox(20,20){}}
\put(10,60){\circle*{7}}
\put(50,50){\framebox(20,20){}}
\put(60,60){\circle*{7}}
\put(20,10){\line(1,0){30}}
\put(20,10){\line(-1,1){10}}
\put(50,10){\line(1,1){10}}
\end{picture}
\caption{The author before embarking on the PhD.}}
\hfill
\parbox{.4\textwidth}{\centering
\begin{picture}(70,70)
\put(0,50){\framebox(20,20){}}
\put(10,60){\circle*{7}}
\put(50,50){\framebox(20,20){}}
\put(60,60){\circle*{7}}
\put(20,10){\line(1,0){30}}
\put(20,10){\line(-1,-1){10}}
\put(50,10){\line(1,-1){10}}
\end{picture}
\caption{The author on completing the PhD.}}
\end{figure}

This is \chapterref{one}.

\section{Intro}\sectionlabel{intro}

Neural networks are cool but mysterious~\cite{lecun2015}.

\section{Notation}\sectionlabel{notation}

Forget \sectionref{intro}.
We need to go over notation,
from $\cA$ to $\cZ$.

A vector is $\vec{x}$.


\onlyinsubfile{\printbibliography}

\end{document}
