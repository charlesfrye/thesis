\documentclass[../../thesis.tex]{subfiles}

% chktex-file 18
\begin{document}

\chapter{Applying Critical Point-Finding Algorithms to Neural Network Loss Functions
Reveals Gradient-Flat Regions}\chapterlabel{three}

\onlyinsubfile{\begin{KeepFromToc}
		\tableofcontents
		\listoffigures
	\end{KeepFromToc}}
\onlyinsubfile{\clearpage}
\onlyinsubfile{\linenumbers}

\section{Chapter Summary}%
\sectionlabel{ch3-summary}

\chapterref{two} motivated and defined
a number of second-order algorithms
for finding the critical points of neural network loss functions.
As described in \chapterref{one},
information about the local curvature at these points is useful
for understanding the optimizability of neural networks.
For example, the No-Bad-Local-Minima theory
(NBLM\@; see \sectionref{nblm}),
based on a random function model of neural network losses
(\sectionref{grf}, \sectionref{wishart}),
predicts that critical points with strictly positive curvature
should only occur when the value of the loss is low.
This implies that first-order methods like gradient descent
can optimize neural networks%
~\cite{jin2018a}.
Previous work%
~\cite{dauphin2014,pennington2017}
appeared to verify this theory.
More recent analytical work, however,
has indicated that the NBLM theory is false,
and there are in fact bad local minima
on neural network loss functions
(\cite{ding2019}; see \sectionref{critic}).
This disagreement with theory
motivates a closer look at the numerical evidence,
which is the substance of this chapter.

First, in \sectionref{dlae},
we present a test problem,
the deep linear autoencoder,
that is analytically tractable
while remaining sufficiently similar
to the analytically intractable problem
of interest, viz.~non-linear networks.
The analytically-derived critical points
are used to verify that the approximate critical points
recovered by the numerical algorithms are accurate.
Then, we apply the best-performing of these methods,
Newton-MR, to a non-linear network
and observe a tremendous change in its behavior:
qualitative signatures of convergence disappear
and quantitative metrics sharply decline
(\sectionref{nonlinear-failure}).

We return, then, to the analysis of second-order
critical point-finding methods and demonstrate
that a large class of spurious targets of convergence
has been over-looked:
points where the gradient lies in the kernel of the Hessian.
We call these \emph{gradient-flat points}.
They are defined in~\sectionref{gfr}.
We present evidence that these points are encountered
by critical point-finding methods applied to neural network losses in%
~\sectionref{gfr-results}.

The gradient-flat points that cause the most trouble are
bad local minima of the gradient norm.
The core result of this chapter
and this dissertation, then,
is that the second-order methods
used to find the critical points
of neural network loss functions
in an attempt to prove the no-bad-local-minima theory
suffer from a bad-local-minima problem of their own.

\section{The Deep Linear Autoencoder Provides a Useful Test Problem}%
\sectionlabel{dlae}

This section introduces the deep linear autoencoder
as a test problem for critical point-finding methods
for neural network loss functions.
First,
the need for test problems is explained
(\sectionref{test}).
Then,
the deep linear autoencoder is presented
and the critical points derived
\sectionref{def-dlae}.
Finally, the performance of
Newton-MR (\algoref{nmr}),
damped Newton (\algoref{damped-newton}),
and gradient norm minimization by
gradient descent with back-tracking line search
(BTLS-GNM)
are compared
(\sectionref{dlae-results}).

Additional results regarding these critical point-finding methods
and the deep linear autoencoder are
published in~\cite{frye2019}.

\subsection{Test Problems are Necessary}%
\sectionlabel{test}

In the case of optimization for the solution of engineering problems,
the question of convergence to the neighborhood of a true minimizer
is often an academic one.
The goal of the optimization procedure is to find a point
at which the loss is sufficiently low to
e.g.~allow the airplane to fly
or select the better ad to display to users.
The problem of finding critical points is different:
our goal is to check analytical properties,
which can in principle require arbitrary precision.

The usual solution is to prove an upper bound on the precision required
by means of inequalities.
In our case, we are interested in two quantities:
the loss $L$ and
the index $I$ (\defref{index})
at critical points $\theta_{\cp}$.
For the loss,
we might proceed by demonstrating that the function is
$K$-Lipschitz for some constant $K$.

\begin{definition}{Lipschitz Continuity}{lipschitz}
	A function $f\from \R^n \to \R$ is said to be \emph{$K$-Lipschitz} if
	\begin{equation}
		\norm{f(x + \eps) - f(x)} \leq K\norm{\eps}
	\end{equation}
	for some $K\in \R_{\geq 0}$ and for all $x,\eps \in \R^n$.
\end{definition}
\noindent For an almost everywhere differentiable function,
like $\ReLU$, $K$-Lipschitz continuity is equivalent to demonstrating
that the gradient norm is bounded by $K$.
Using this fact,
a bound on the difference between the loss
at an approximate critical point with gradient norm $\eps$
and any nearby true critical points can be derived.

However, the Lipschitz constants of typical neural network
losses are not well-controlled
\cite{gouk2018}, %chktex 2
resulting in worst-case bounds that are overly-pessimistic.
The situation is even worse for calculating the index,
which depends both on Lipschitz bounds on the operator norm
of the Hessian and on guarantees that the kernel changes only slowly.

In the absence of these guarantees,
the results of numerical algorithms must be interpreted with care
and hyperparameters must be tuned cautiously.
The behavior of algorithms during convergence should be closely monitored
and reported along with results.
And critically, it is important to test numerical algorithms
on problems for which the answers are known
before applying them to problems for which the answers are in doubt.
In addition to allowing for hyperparameters to be set to reasonable starting values
and for the signatures of convergence to be identified,
test problems provide an important opportunity to debug implementations
that is missing in cases without ground truth.
Analytical guarantees are of little value
if the algorithms in question are broken.

Furthermore, there are many critical point-finding algorithms that might be chosen.
\chapterref{two} presented two Newton methods
of sufficient robustness for practical use,
plus gradient norm minimization.
Both the damped Newton method~\cite{dauphin2014}
and gradient norm minimization~\cite{pennington2017}
have been used,
but sufficient information for comparing performance
was not published.

It is critical that the test problem be designed
to be as close as possible to the problem of interest.
The following section introduces such a test problem for
neural network loss functions.

\subsection{The Deep Linear Autoencoder has Known Critical Points}%
\sectionlabel{def-dlae}

The key difficulty for deriving analytical expressions
for the critical points of non-linear neural networks
is that the gradients of the loss are non-linear
in an arbitrary fashion.
This turns the problem of finding critical points
into the problem of finding solutions to generic non-linear equations.

Neural networks without non-linearities,
also known as \emph{linear networks},
avoid this problem.
The network is constructed by raveling the entries of $\theta$
into a $D$-element sequence of matrices $W_i(\theta)$
which multiply an input $x$ in turn.
We call these matrices the \emph{weight matrices}
of the network and say that the network has $D$ layers,
each with \emph{layer size} equal to the number of rows
in the matrix $W_i$.
The layers $W_1$ through $W_{D-1}$ are called \emph{hidden layers}.
\begin{align}
	\mathrm{NN}_{\mathrm{DLN}}(\theta)(x)
	&= W_D(\theta) W_{D-1}(\theta) \dots W_{2}(\theta) W_{1}(\theta) x\\
	&\defeq{W(\theta) x}
\end{align}
\noindent As the second line indicates,
this is equivalent to applying
a single matrix $W(\theta)$.
Therefore the function computed by the network as a whole is linear.
Any non-linearity in the gradient function
of the loss of such a network is introduced
only by the cost function and regularizer.
For simplicity, we consider only the unregularized case.

The loss of this network can be construed as a function
of any of three quantities.
As before, it is a function of the vector of parameters $\theta$,
which we continute to denote $L$.
It is also a function of the separate weights of each layer,
the matrices $W_i$,
which we denote $l$
and refer to as the \emph{layerwise} loss.
These functions also exist for non-linear networks.
Finally, in the linear case
the loss can also be written a function
of the \enquote{collapsed} matrix $W$,
which we denote $\cL$
and refer to as the \emph{collapsed} loss.
We further drop the explicit dependence of $W$
and the $W_i$ on $\theta$ when writing the loss in this way.

With this setup,
we can write the gradients of a linear network
of arbitrary depth
with arbitrary cost in simple form.
\begin{theorem}{Gradients of a Deep Linear Network}{dln-grad}
	\emph{%
	Let $l$ be the layerwise loss
	and $\cL$ the collapsed loss
	of a deep linear network
	with $D$ layers.
	The gradient function of the layerwise loss
	with respect to layer $k$,
	denoted $\nabla_{W_k}(l)$, is
	\begin{equation}
		\grad{_{W_k} l}{W_1, \dots, W_D}
		= W_{k+1:}^\top \grad{\cL}{W} W_{:k}^\top\equationlabel{def-dln-grad}
	\end{equation}
	}
\end{theorem}
\noindent where the notation $W_{i:j}$,
inspired by the slicing syntax of Python and other languages,
stands for the products of matrices $i$ to $j-1$,
with an empty entry on the left standing for $1$
and an empty entry on the right standing for $D+1$.

\noindent \emph{Proof of \thmref{dln-grad}}:\\
This proof follows closely that in%
~\cite{laurent2018}.
The gradient is defined in terms of the value
of the function at an input perturbed by $\eps$
\begin{equation}
	l(W_1, \dots, W_{k-1}, W_k + \eps, W_{k+1}, \dots W_D)
\end{equation}
\noindent where here $\eps$ is a matrix of the same shape as $W_k$.
We proceed by converting to the collasped loss $\cL$
and multiplying through the matrix product.
\begin{align}
	l(W_1, \dots, W_{k-1}, W_k + \eps, W_{k+1}, \dots W_D)
	&= \cL\left(W_1 W_2 \dots W_{k-1} (W_k + \eps) W_{k+1} \dots W_D\right)\\
	&= \cL\left(W + W_{k+1:} \eps W_{:k}\right)\\
	&= \grad{\cL}{W} + \langle \grad{\cL}{W}, W_{k+1:} \eps W_{:k}\rangle + o(\eps)
	\equationlabel{taylor-collapsed-loss}
\end{align}
\noindent where the last line follows by pattern-matching
to the definition of the gradient function,
\defref{gradient}.
Note that the inner product here is an inner product of matrices.
It is the Frobenius inner product
\begin{equation}
	\langle A, B \rangle = \tr\left(A^\top B\right)
\end{equation}
\noindent which is defined by unraveling the matrices into vectors
and applying the Euclidean inner product of vectors,
i.e.~as a pullback of that inner product via a raveling isomorphism.
The trace $\tr$ is invariant to cyclic permutations,
and so we can re-organize the middle term of
\equationref{taylor-collapsed-loss}
\begin{align}
	\langle \grad{\cL}{W}, W_{k+1:} \eps W_{:k}\rangle
	&= \tr\left(\grad{\cL}{W}^\top W_{k+1:} \eps W_{:k} \right)\\
	&= \tr\left(W_{:k} \grad{\cL}{W}^\top W_{k+1:} \eps \right)\\
	&= \langle W_{k+1:}^\top \grad{\cL}{W}W_{:k}^\top, \eps\rangle
\end{align}
\noindent which implies, by pattern-matching
to the definition of the gradient function again,
\begin{equation}
	\grad{_{W_k} l}{W_1, \dots, W_D}
	= W_{k+1:}^\top \grad{\cL}{W} W_{:k}^\top
\end{equation}
\QED\\

This suggests that if we wish to be able to
come up with an analytic expression for the critical points,
which make this gradient function $0$,
we choose a collapsed loss function $\cL$
that has a simple form that allows us to set
\equationref{def-dln-grad} equal to $0$ and solve.

We therefore make several simplifying choices
in constructing our test problem.
First, we take the inputs and targets
to be the same.
This type of network is known as an \emph{autoencoder}.
Second, we choose the squared error as the loss function.
Finally, we choose the number of layers $D$ to be $2$.
We refer to this combination as the \emph{deep linear autoencoder}.

The critical points of this network can be characterized as follows:
\begin{theorem}{Critical Points of Deep Linear Autoencoder}{dlae-cps}
	\emph{%
	Let $L$ be the loss function of a linear network such that
	\begin{equation}
		L(\theta) = \snorm{X - W_2(\theta)W_1(\theta) X}
	\end{equation}
	\noindent for some matrix $X\in\R^{k\times n}$
	such that $XX^\top$ is full rank,
	with $W_1\from \R^{2kp}\to \R^{p\times k}$
	$W_2\from \R^{2kp}\to \R^{k \times p}$.
	\\ \ \\
	Then the critical points of this network
	correspond to those $\theta_{\cp}$
	such that the matrix
	$W = W_2(\theta_\cp)W_1(\theta_\cp)$
	acts as the identity on a subspace
	spanned by some subset of the eigenvectors of $XX^\top$
	and the zero matrix otherwise.
	}
\end{theorem}

Before diving into the proof,
we first consider the interpretation and implications.
The matrix $XX^\top$ is the
sample covariance matrix of the data $X$
when the data has sample mean $0$.
The eigenvectors of this matrix are known as the
\emph{principal components}
of the data.
A single critical point can be constructed
according to the criterion in the theorem above
by choosing $m \leq p$ separate principal components
for the rows of $W_1$
(setting the others to $0$ when $m<p$)
and choosing $W_2 = W_1^\top$.
This allows the construction of
a number of distinct critical points equal to the number of ways
to choose from $0$ to $p$ elements
from a $k$ element set:
$\sum_{i=0}^p \binom{k}{i}$.
When the eigenvectors selected are the $m$
with largest eigenvalue,
this network is performing \emph{principal components analysis}.
This value corresponds to the global minimum.
It can be demonstrated that there are no non-global local minima%
~\cite{baldi1989,laurent2018},
and so this loss function has the no-bad-local-minima property.
It is a generalization of~\exampleref{1ddlae}.

An uncountable collection of additional critical points
can be constructed from the points corresponding directly
to the principal components.
Indeed, the key property in~\thmref{dlae-cps}
is defined in terms of the collapsed matrix $W$.
Applying any invertible $p\times p$ matrix $C$ after $W_1$
and its inverse $C^{-1}$ before $W_2$
leaves $W$ unchanged\footnote{%
Note that invertible matrices form a Lie group,
and therefore so do these critical points.
This situation is shared, to an extent,
in $\ReLU$ networks~\cite{freeman2016}.},
and so the resulting new point is also a critical point.
The loss and index are unchanged,
and so we can consider these collections
to be equivalence classes of critical points,
with representatives given by the choice $C=C^{-1}=I$.

\thmref{dlae-cps},
which was first proven in%
~\cite{baldi1989},
is proven in the following section
for completeness.
Readers uninterested in the technical details
are invited to skip to
\sectionref{dlae-results}.

\subsubsection{Proof of \thmref{dlae-cps}}%
\sectionlabel{dlae-proof}

The collapsed loss function of the linear network is
\begin{equation}
	\cL(W) = \snorm{X - WX}
\end{equation}
\noindent Expanding the right-hand side,
we have that
\begin{align}
	\cL(W + \eps)
	&= \tr\left(X^\top X\right) - \tr\left(2X^\top(W+\eps)X\right)
	+ \tr\left(X^\top{(W+\eps)}^\top(W+\eps)X\right)\\
	&= \cL(W) + \tr\left(2 XX^\top W \eps - 2XX^\top\eps\right) + o(\eps)
\end{align}
\noindent which, after re-arrangement, gives the gradient function
for $\cL$ by pattern-matching to \defref{gradient}:
\begin{equation}
	\grad{\cL}{W} = 2 XX^\top (W - I)
\end{equation}

To find the analytical critical points,
we plug this definition into
the expressions for the gradients with respect to
the two weight matrices $W_1$ and $W_2$
given by \thmref{dln-grad}
and solve for 0.

We start with $W_2$:
\begin{align}
	\grad{l_{W_2}}{W_1, W_2} &= \grad{\cL}{W} W_1^\top\\
	0 &= \grad{\cL}{W} W_1^\top\\
	0 &= 2 XX^\top\left(W_2 W_1 - I\right) W_1^\top\\
	W_1^\top &= W_2W_1 W_1^\top
\end{align}
\noindent where the transition to the third line
used the invertibility of $XX^\top$.
This equation is satisfied
whenever $W_2 W_1$ is equivalent to the identity
matrix when restricted to the co-kernel of $W_1$,
which is the range of $W_1^\top$.

This isn't sufficient to completely determine
$W_1$ and $W_2$,
so we proceed to the equations
for $W_1$:
\begin{align}
	0 &= W_2^\top\grad{\cL}{W}\\
	0 &= W_2^\top 2 XX^\top\left(W_2 W_1 - I\right)\\
	W_2^\top &= W_2^\top XX^\top W_2 W_1 {\left(X X^\top\right)}^{-1}
\end{align}
\noindent When the matrix $XX^\top$
is simultaneously diagonalizable with the matrix $W_2W_1$,
the matrices commute,
which gives the equation
\begin{align}
	W_2^\top &= W_2^\top W_2 W_1
\end{align}
\noindent
This equation holds when $W_2 W_1$ is equivalent to the identity
when applied to the co-kernel of $W_2^\top$.

For $W_2 W_1$ to be simultaneously diagonalizable with $XX^\top$,
the non-zero eigenvectors of $W_2W_1$
need to be the same as some subset of the eigenvectors of $XX^\top$.
Together, these conditions imply that
at a critical point of the loss,
$W_2W_1$ acts as the identity
on a subspace spanned by some subset of the eigenvectors of $XX^\top$.

\subsection{Newton-MR Outperforms Previous Methods on this Problem}%
\sectionlabel{dlae-results}

We are now ready to set up the deep linear autoencoder
test problem and compare the performance of our
critical point-finding algorithms.

\subsubsection{Data and Network}\sectionlabel{dlae-parameters}

\thmref{dlae-cps} gives a characterization
of the set of all critical points, $\Theta_{\cp}$,
for a given choice of dataset $X\in\R^{k \times n}$ and two-layer network.
The key parameters for the dataset are its covariance matrix,
dimension $k$, and size $n$.
The key architectural hyperparameter of the network is the
size of the hidden layer
($p$, as in \thmref{dlae-cps}).

For simplicity, we choose a multivariate Gaussian distribution
for the columns of $X$,
as this distribution is entirely specified by its mean and covariance.
The connection to PCA
required that $XX^\top$ be the sample covariance matrix of $X$,
and so we enforce that $X$ has row means exactly $0$.
Note that this must be done after sampling,
as setting the distributional means to $0$ does not
result in sample row means of $0$.
We choose a diagonal covariance matrix.
To obtain maximally-spaced eigenvalues while still
controlling the condition number of $XX^\top$,
we space the diagonal values linearly between $1$ and $k$,
inclusive.
We choose $n$, the number of samples, to be $1e4$.

The parameters $k$, for the dimension of the data,
and $p$, for the dimension of the hidden layer of the neural network,
directly determine the number of critical points,
up to equivalence.
Since this count is effectively
given by the number of ways to choose subsets from a set of size $k$,
it grows nearly as fast as $2^k$.
For this reason, we choose $k$ to be small, relative to the typical
inputs to neural networks: $16$, as opposed to order $100$ or $1000$.
Selecting $p$ also to be small, specifically $4$,
gives a total number of equivalence classes of critical points equal to
$2517 = \sum_{i=0}^4 \binom{16}{i}$.
Below, we will call the set of all representatives of the
equivalence classes the \emph{analytical critical points}
of the deep linear autoencoder.

With all of these choices in place,
we can now calculate the loss and index values of the analytical
critical points for a neural network loss surface.
These are plotted in \figureref{dlae-cps}.
The squared gradient norms of these points are generally not $0$,
but on the order of 1e-32.
Already, this provides a loose lower bound on the precision we need
in order to recover critical points.
There is one notable exception:
the $0$ vector, which is a critical point for this loss.
Here, the gradient is exactly $0$.
In general, the density of floating point numbers is greatest near $0$,
and so critical points near $0$ can be recovered with greater accuracy.
Note that, in a numerical setting,
the index is defined as the number of eigenvalues
below some small negative value, rather than $0$,
to account for error.
We choose -1e-5,
based on inspection of the spectra of these
analytical critical points.

\begin{figure}[h]
	\begin{center}
		\includegraphics[width=0.5\textwidth]{img/chapter3/dlae-cps.pdf}
	\end{center}
	\caption{\textbf{The Analytical Critical Points
	of a Deep Linear Autoencoder}}
	\figurelabel{dlae-cps}
\end{figure}

\subsubsection{Two-Step Process for Sampling Critical Points}%
\sectionlabel{dlae-sampling}

The full set of critical points, $\Theta_{\cp}$,
is uncountably infinite in size,
as is the full set of $\eps$-critical points,
$\Theta_{\ecp}$.
Even the set of critical points up to equivalence is exponentially large.
For a problem as small as this one, it might be feasible to find all of them,
but for larger networks this won't be the case.
Instead, our methods must aim to sample the set of critical points.
If the sample is unbiased, qualitative and quantitative features
of the loss-index relationship
(\figureref{dlae-cps})
should be preserved.

To obtain initial points for our critical point-finding methods,
we first apply an optimization algorithm to the loss.
This generates a trajectory of values of $\theta$ with varying loss,
from near the value of the highest critical point
to near the value of the global minimum.
These values are then chosen as initial points
for critical point-finding algorithms.
We follow~\cite{pennington2017} in selecting these points
uniformly at random based on their loss value.
Due to the exponential rate of first-order optimization algorithms,
selecting points uniformly at random from the trajectory
would over-represent lower loss values.
Furthermore, decreasing gradient norms during training mean that
later parameter values are closer together,
and so this scheme would heavily over-sample a small region.
Both factors should reduce the variety in recovered critical points%
\footnote{These approaches were directly compared in
experiments not presented here and this intuition is borne out.
See~\cite{frye2019}.}.
To half of the initial points attained in this manner,
we add Gaussian noise of with variance 1e-2
(an SNR of approximately 2.8 dB).
In separate experiments,
noise of this magnitude
this was found to increase
the diversity in critical points recovered%
~\cite{frye2019}.

\subsubsection{Critical Point-Finding Methods Have Differential Performance}%
\sectionlabel{dlae-perf-comparison}

\begin{figure}[h]
	\begin{center}
		\includegraphics[width=\textwidth]{img/chapter3/perf-comparison.pdf}
	\end{center}
	\caption{\textbf{Newton-MR, Damped Newton, and BTLS-Gradient Norm Minimization
	can Recover Critical Points of a Deep Linear Autoencoder}.}{%
		Horizontal and vertical axes are shared across
		the top row and the bottom row of panels separately.
	}
	\figurelabel{perf-comparison}
\end{figure}

\subsubsection{Cutoffs Must be Set Strictly}%
\sectionlabel{dlae-cutoffs}

As noted in \sectionref{test},
one of the key purposes of applying our algorithms
to a test problem is to obtain estimates for the value
of $\eps$ necessary to guarantee that the loss and index values
for members of $\Theta^L_{\ecp}$
are similar to those of $\Theta^L_{\cp}$.
With such a value in hand,
we can terminate our critical point-finding algorithms early,
as soon as they produce a candidate critical point with
squared gradient norm below $\eps$.

We simulate the results of such a procedure
by applying a \enquote{cutoff}
to the traces from \figureref{perf-comparison}A.
We truncate each run at the first $T$ such that
$\sgn{L}{\theta_T} < \eps$.
As our cutoffs,
we choose the value used to filter the traces in
\figureref{perf-comparison}, 1e-10,
the value used as a filter in~\cite{pennington2017}, 1e-6,
a looser criterion of 1e-4,
and $\infty$.
The latter corresponds to no cutoff,
taking the initial points,
the iterates of the optimization algorithm,
as candidate critical points.
The results are in \figureref{cutoffs}.

\begin{figure}[h]
	\begin{center}
		\includegraphics[width=0.7\textwidth]{img/chapter3/cutoffs.pdf}
	\end{center}
	\caption{\textbf{Cutoffs Above 1e-10 are Insufficient to Guarantee
	Accurate Loss and Index Recovery}.}{%
	As in \figureref{perf-comparison},
	$\eps$-CPs are plotted in blue
	over analytical CPs in gray.
	For each panel,
	$\eps$-CPs are selected by taking
	the 100 runs of Newton-MR in \figureref{perf-comparison}A
	and taking the first point whose squared gradient norm is below
	the cutoff value, $\eps$, in the top-left corner.
	Horizontal and vertical axes are shared between panels.
}
	\figurelabel{cutoffs}
\end{figure}

Notably, using the value of 1e-10
that, as a filter,
resulted in accurate recovery of loss and index values
(\figureref{perf-comparison}),
introduces a small amount of error,
in particular at low and high values of the index
(\figureref{cutoffs}A).
This suggests that
a strategy of running
critical point-finding methods until termination
and then filtering
gives better results than
a strategy of early termination.
Looser cutoff values
(\figureref{cutoffs}B,C)
result in quantitatively worse
recovery of loss-index values
but qualitatively similar loss-index relationships.
Interestingly, applying no cutoff at all,
as in \figureref{cutoffs}D,
results in a convex upwards relationship
between index and loss,
much like that reported in~\cite{dauphin2014} and~\cite{pennington2017}
for non-linear networks,
despite the fact that the true relationship is linear.


\section{Methods that Work on a Linear Network Fail on a Non-Linear Network}%
\sectionlabel{nonlinear-failure}

\begin{figure}[h]
	\begin{center}
		\includegraphics[width=0.5\textwidth]{img/chapter3/nonlinear-failure.pdf}
	\end{center}
	\caption{\textbf{Newton-MR Fails to Find Critical Points
	on a Non-Linear Network}}
	\figurelabel{nonlinear-failure}
\end{figure}

\section{Gradient-Flat Regions can Cause Critical Point-Finding Methods to Fail}%
\sectionlabel{gfr}

In this section,
we introduce and define gradient-flat points and
explain why they are problematic for second-order
critical point-finding methods
\sectionref{theory},
with the help of a low-dimensional example
to build intuition
\sectionref{toy}.
In numerical settings and in high dimensions,
approximately gradient-flat points are also important,
and so we define a quantitative index of gradient-flatness
based on the residual norm of the Newton update
\sectionref{approxgfp}.
Connected sets of these numerically gradient-flat points
are gradient-flat regions,
which cause trouble for
second-order critical point-finding methods.

\subsection{At Gradient-Flat Points, the Gradient Lies in the Hessian's Kernel}%
\sectionlabel{theory}

Critical points are of interest because
they are points where the first-order approximation
of a function  $f$ at a point
$x+\delta$ based on the local information at $x$
\begin{equation}
    f(x + \delta) \approx f(x) + \grad{f}{x}^\top \delta
\end{equation}
is constant, indicating that they are the stationary points
of first-order optimization algorithms
like gradient descent and its variants.

We can similarly understand the behavior of our second-order
critical point-finding methods by considering their stationary points.
In our construction of these methods in
\chapterref{two},
we used a linear approximation of the behavior
of the gradient function
at a point $x + p$ given the local information at a point $x$
\begin{equation}
    \grad{f}{x + p} \approx \grad{f}{x} + \hess{f}{x}p
\end{equation}

Solving for the value of $p$
that makes the left-hand side $0$
gave us the Newton update
\begin{equation}
	\pinv{\hess{f}{x}}\grad{f}{x}
\end{equation}
\noindent which is only $0$,
for a non-singular Hessian,
if $\grad{f}{x}$ is $0$.
For a singular Hessian,
the update $p$ is zero iff $\grad{f}{x}$ is
in the kernel of the pseudoinverse.

In gradient norm minimization,
we search for critical points
by applying first-order optimization methods
to the squared gradient norm function,
$g(\theta) = \sgn{L}{\theta}$.
These methods also make a linear approximation
of the gradient function of $L$.
The gradient function of $g$ is
\begin{equation}
	\grad{g}{x} = \hess{f}{x}\grad{f}{x}
\end{equation}
\noindent As with Newton methods, in the invertible case
the updates are zero iff $\grad{f}{x}$ is 0.
In the singular case,
the updates are zero if the gradient is in the Hessian's kernel.

In the case of invertible Hessians, then,
these methods can guarantee convergence to critical points.
However, neural network Hessians are generally singular,
especially in the overparameterized case~\cite{sagun2017,ghorbani2019}.
In this case,
neither class of methods can guarantee
convergence to critical points
(for GNM, see~\cite{doye2002};
for Newton methods, see~\cite{powell1970,griewank1983}).

What are the stationary points, besides critical points,
for these two method classes in the case of singular Hessians?
It would seem at first that they are different:
for gradient norm minimization,
when the gradient is in the Hessian's kernel;
for Newton-type methods,
when the gradient is in the Hessian's pseudoinverse's kernel.
In fact, however,
these conditions are identical, due to the Hessian's symmetry,
and so both algorithms share a broad class of stationary points.
A proof, based on the construction of the pseudo-inverse in
\sectionref{pinv-svd}, follows.

\begin{theorem}{Kernel Equals Pseudo-Inverse Kernel for Symmetric $M$}%
{kernel-pinvkernel}
	\emph{%
		Let $M\in\R^{n \times n}$ be a symmetric matrix.
		Then
		\begin{equation}
			\ker M  = \ker\pinv{M}
		\end{equation}
	}
\end{theorem}

\noindent \textit{Proof of \thmref{kernel-pinvkernel}}:\\
We first repeat the commutative diagram
relating the SVDs of a matrix and its pseudo-inverse,
specialized to a square matrix.

\begin{center}
    \begin{tikzcd}
	    \R^n \arrow[rr, "M", bend left] \arrow[dd, "V^\top", swap, two heads, bend right]
	    &  & \R^n \arrow[dd, "U^\top", swap, two heads, bend right] \arrow[ll, "M^+", bend left]                       \\
	\ & \  & \\
	\co \ker M \arrow[uu, "V", harpoon, hook, bend right]  \arrow[rr, "\Sigma", bend right]
	&  & \im M \arrow[uu, "U", harpoon, hook, bend right] \arrow[ll, "\inv{\Sigma}", bend right]
    \end{tikzcd}
\end{center}

Note that $\im M$ is also $\co\ker \pinv{M}$,
by reading the diagram counter-clockwise.
The SVD of the transpose of $M$
can be attained by transposing each of the elements of the SVD\@:
\begin{align}
	M^\top
	&= {\left(U \Sigma V^\top\right)}^\top\\
	&= V \Sigma U^\top
\end{align}

This can be summarized in the commutative diagram below.

\begin{center}
    \begin{tikzcd}
	    \R^n \arrow[rr, "M", bend left] \arrow[dd, "V^\top", swap, two heads, bend right]
	    &  & \R^n \arrow[dd, "U^\top", swap, two heads, bend right] \arrow[ll, "M^\top", bend left]                       \\
	\ & \  & \\
	\co \ker M \arrow[uu, "V", harpoon, hook, bend right]  \arrow[rr, "\Sigma", bend right]
	&  & \im M \arrow[uu, "U", harpoon, hook, bend right] \arrow[ll, "\Sigma", bend right]
    \end{tikzcd}
\end{center}

This implies that $\im M = \co\ker M^\top$.
But for a symmetric matrix,
$M = M^\top$,
and so $\co\ker M = \co\ker M^\top = \im M$.
But above, we saw that $\im M = \co\ker \pinv{M}$,
and therefore $\co\ker M = \co\ker \pinv{M}$,
which further implies that $\ker M = \ker \pinv{M}$.
\[\QED\]

These stationary points have been identified previously,
but nomenclature is not standard:
Doye and Wales, studying gradient norm minimization,
call them \emph{non-stationary points}~\cite{doye2002},
since they are non-stationary with respect to the function $f$,
while Byrd et al., studying Newton methods,
call them \emph{stationary points}~\cite{byrd2004},
since they are stationary with respect to the merit function $g$.
To avoid confusion between these incommensurate conventions
or with the stationary points of the function $f$,
we introduce our own terminology.

\begin{definition}{Gradient-Flat Points}{gfp}
	A point $\theta\in\R^n$ is a \emph{gradient-flat} point
	of a twice continuously-differentable function
	$f\from\R^n\to\R$ if
	\begin{equation}
		\grad{f}{\theta} \in \ker \hess{f}{\theta}
	\end{equation}
\end{definition}

This name was chosen because a function is \emph{flat}
when its Hessian is 0, meaning every direction is in the kernel,
and so it is locally flat around a point in a given direction
whenever that direction is in the kernel of the Hessian at that point.
Note that, because $0 \in \ker$ for all matrices,
every critical point is also a gradient-flat point,
but the reverse is not true.
When we wish to explicitly refer to gradient-flat points
which are not critical points,
we will call them \emph{strict} gradient-flat points.
At a strict gradient-flat point, the function is,
along the direction of the gradient,
locally linear up to second order.

There is an alternative view of gradient-flat points
based on the squared gradient norm function $g$.
All gradient-flat points are stationary points
of the gradient norm:
they are zeroes for $\nabla g$.
They may in principle be local minima, maxima, or saddles,
while the global minima of the gradient norm are critical points.
When they are local minima of the gradient norm,
they can be targets of convergence
for methods that use
first-order approximations of the gradient map,
as in gradient norm minimization and in Newton-type methods.
Strict gradient-flat points, then,
can be bad local minima of the gradient norm,
and therefore prevent the convergence of
second-order root-finding methods
to critical points,
just as bad local minima of the loss function
can prevent convergence of first-order optimization methods
to global optima.

Note that Newton methods cannot be demonstrated to converge
only to gradient-flat points~\cite{powell1970,byrd2004}.
Furthermore, Newton convergence can be substantially slowed
when even a small fraction of the gradient
is in the kernel~\cite{griewank1983}.
Below we will see that,
while Newton-MR applied to a neural network loss
sometimes converges to and almost always encounters strict gradient-flat points,
the final iterate is not always either
a strict gradient-flat point or a critical point.

\subsection{Convergence to Gradient-Flat Points in a Quartic Example}%
\sectionlabel{toy}

The difficulties that gradient-flat points pose for Newton methods
can be demonstrated with a polynomial example in two dimensions,
plotted in \figureref{toy}A.
Below,
we will characterize
the strict gradient-flat (\failcolor{})
and critical (\successcolor{}) points of this function
(\figureref{toy}A).
Then, we will observe the behavior of Newton-MR
when applied to it (\figureref{toy}B)
and note similarities to the results in \figureref{nonlinear-failure}.
We will use this simple, low-dimensional example
to demonstrate principles useful
for understanding the results of applying
second-order critical point-finding methods to more complex,
higher-dimensional neural network losses.

\begin{example}{A Quartic Polynomial with a Gradient-Flat Point}{toy}
	\begin{equation}\equationlabel{toy}
	    f(x, y) = 1/4 x^4 - 3x^2 + 9x + 0.9y^4 + 5y^2 + 40
	\end{equation}
\end{example}

\begin{figure}
	\includegraphics[width=0.9\linewidth]{img/chapter3/toy-problem.pdf}
	\caption{\textbf{Stationarity of and Convergence to
	a Strict Gradient-Flat Point on a Quartic Function}.}%
	{\textbf{A}:
	Critical and strict gradient-flat points
	of quartic $f(x,y)$ (defined in \equationref{toy}).
	Central panel:
	$f(x,y)$
	plotted in black and white
	(black, low values; white, high values),
	along with the direction of the Newton update $p$
	as a (notably non-smooth) vector field (red).
	Stationary points of
	the squared gradient norm merit function $g$ are indicated:
	strict gradient-flat points in \failcolor{},
	the critical point in \successcolor{}.
	Top and bottom panels:
	The value (top) and squared gradient norm (bottom)
	of $f$ as a function of $x$ value
	with $y$ fixed at 0.
	The $x$ axis is shared between panels.
	\textbf{B}:
	Performance and trajectories of Newton-MR
	on~\equationref{toy}.
	Runs that terminate near a strict gradient-flat point
	are in \failcolor{},
	while those that terminate near
	a critical point are in \successcolor{}.
	Central panel:
	Trajectories of Newton-MR laid over
	$f(x, y)$.
	$x$ and $y$ axes are shared with the central panel of
	\emph{A}.
	Initial values indicated with scatter points.
	Top and bottom panels:
	Function values (top) and squared gradient norms (bottom)
	of Newton-MR trajectories as a function of iteration.
	The $x$ axis is shared between panels.}\figurelabel{toy}
\end{figure}

\equationref{toy} is plotted in~\figureref{toy}A,
central panel.
This quartic function has two affine subspaces
of points with non-trivial Hessian kernel,
defined by $[\pm\sqrt{2}, y]$.
The kernel points along the $x$ direction and so
is orthogonal to this affine subspace at every point.
As a function of $y$, $f$ is convex,
with one-dimensional minimizers at $y=0$.
The strict gradient-flat points occur at the intersections
of these two sets:
one strict gradient-flat point at $[\sqrt{2}, 0]$,
which is a local minimum of the gradient norm,
and one at $[-\sqrt{2}, 0]$,
which is a saddle of the same
(\figureref{toy}A, \failcolor{} points, all panels).
In the vicinity of these points, the gradient is,
to first order, constant along the $x$-axis,
and so the function is locally linear or flat.
These points are gradient-flat but
neither is a critical point of $f$.
The only critical point is located at the minimum of the polynomial,
at $[-3, 0]$
(\figureref{toy}A, \successcolor{} point, all panels),
which is also a global minimum of the gradient norm.
The affine subspace that passes through
$[-\sqrt{2}, 0]$ divides the space into two
basins of attraction, loosely defined,
for second-order methods:
one, with initial $x$-coordinate $x_0<-\sqrt{2}$,
for the critical point of $f$
and the other for the strict gradient-flat point.
Note that the vector field in the central panel shows update directions
for the pure Newton method,
which can behave extremely poorly in the vicinity
of singularities~\cite{powell1970,griewank1983},
often oscillating and converging very slowly
or diverging.

Practical Newton methods,
like those introduced in \chapterref{two}
and used in \sectionref{dlae-results} above,
use techniques like
damping and line search to improve behavior.
To demonstrate how a practical Newton method
behaves on this function,
we focus on the case of Newton-MR\@.
Results are qualitatively similar for damped Newton.

The results of applying Newton-MR
to~\equationref{toy} are shown in%
~\figureref{toy}B.
The gradient-flat point is attracting
for some trajectories
(\failcolor{}),
while the critical point is attracting for others
(\successcolor{}).
For trajectories that approach the strict gradient-flat point,
the gradient norm does not converge to 0,
but converges to a non-zero value near 10
(\failcolor{} trajectories; \figureref{toy}B, bottom panel).
This value is typically several orders of magnitude lower
than the initial point, and so would appear to be close to 0
on a linear scale that includes the gradient norm of the initial point.
Since log-scaling of loss functions is uncommon in machine learning,
as losses do not always have minima at 0,
second-order methods apporaching gradient-flat points
can appear to converge to critical points
if typical methods for visually assessing convergence are used.

There are two interesting and atypical behaviors worth noting.
First, the trajectories tend to oscillate
in the vicinity of the gradient-flat point
and converge more slowly
(\figureref{toy}B, central panel, \failcolor{} lines).
Updates from points close to the affine subspace where the Hessian has a kernel,
and so which have an approximate kernel themselves,
sometimes jump to points where the Hessian doesn't have an approximate kernel.
This suggests that, when converging towards a gradient-flat point,
the degree of flatness will change iteration by iteration.
Second, some trajectories begin in the nominal basin
of attraction of the gradient-flat point
but converge to the critical point
(\figureref{toy}B, central panel, \successcolor{} points
with $x$-coordinate $>-\sqrt{2}$).
This is because the combination of back-tracking line search
and large proposed step sizes means that occasionally,
very large steps can be taken, based on non-local features of the function.
Indeed,
back-tracking line search is a limited form of global optimization
and the ability of line searches
to change convergence behaviors predicted from local properties
on nonconvex problems
is known~\cite{nocedal2006}.
Since the back-tracking line search is based on the gradient norm,
the basin of attraction for the true critical point,
which has a lower gradient norm than the gradient-flat point,
is much enlarged relative to that for the gradient-flat point.
This suggests that Newton methods
using the gradient norm merit function will be biased
towards finding gradient-flat points
that also have low gradient norm.

\subsection{Approximate Gradient-Flat Points Form Gradient-Flat Regions}%
\sectionlabel{approxgfp}
Analytical arguments focus on exactly gradient-flat points,
where the Hessian has an exact kernel
and the gradient is entirely within it.
In numerical settings,
it is almost certain no matrix will have an exact kernel,
due to rounding error.
For the same reason, the computed gradient vector will generically not lie entirely
within the exact or approximate kernel.
However, numerical implementations of second-order methods
will struggle even when there is no exact kernel
or when the gradient is only partly in it,
and so a numerical index of flatness is required.
This is analogous to the requirement to specify a tolerance
for the norm of the gradient when deciding whether to consider a point
an approximate critical point or not.

We quantify the degree of gradient-flatness of a point
by means of
the \emph{relative residual norm} ($r$)
and the \emph{relative co-kernel residual norm} ($r_H$)
for the Newton update direction $p$.
The vector $p$ is an inexact solution to the Newton system $Hp + g = 0$,
where $H$ and $g$ are the current iterate's Hessian and gradient.
The residual is equal to $Hp + g$,
and the smaller its norm, the better $p$ is as a solution.
The co-kernel residual is equal to the Hessian times the residual,
and so ignores any component in the kernel of the Hessian.
Its norm quantifies the quality of an inexact Newton solution
in the case that the gradient lies partly in the Hessian kernel,
the unsatisfiable case, where $Hp \neq -g$ for any $p$.
When the residual is large but the co-kernel residual is small
(norms near 1 and 0, respectively, following suitable normalization),
then we are at a point where
the gradient is almost entirely in the kernel of the Hessian:
an approximate gradient-flat point.
In the results below, we consider a point approximately gradient-flat
when the value of $r_H$ is below 5e-4
while the value of $r$ is above $0.9$.
We emphasize that numerical issues for second-order methods
can arise even when the degree of gradient-flatness
is small.

The relative residual norm, $r$,
measures the size of the error
of an approximate solution to the Newton system:
\begin{equation}
    r(p) = \frac{\norm{Hp + g}}{\norm{H}_F \norm{p} + \norm{g}}
\end{equation}
\noindent where $\norm{M}_F$ of a matrix $M$ is its Frobenius norm.
Since all quantities are non-negative,
$r$ is non-negative;
because the denominator bounds the numerator,
by the triangle inequality and the compatibility
of the Frobenius and Euclidean norms,
$r$ is at most 1.
For an exact solution of the Newton system $p^\star$,
$r(p^\star)$ is 0, the minimum value,
while $r(0)$ is 1, the maximum value.
Note that small values of $\norm{p}$
do not imply large values of this quantity,
since $\norm{p}$ goes to 0 when a Newton method
converges towards a critical point,
while $r$ goes to 0.

When $g$ is partially in the kernel of $H$,
the Newton system is unsatisfiable,
as $g$ will also be partly in the co-image of $H$,
the linear subspace into which $H$ cannot map any vector.
In this case, the minimal value for $r$ will no longer be $0$.
The optimal solution for $\norm{Hp + g}$
instead has the property that its residual
is $0$ once restricted to the co-kernel of $H$.
This co-kernel residual can be measured by applying the matrix $H$
to the residual vector $Hp + g$.
After normalization, it becomes
\begin{equation}
    r_H(p) = \frac{\norm{H(Hp + g)}}{\norm{H}_F \norm{Hp + g}}
\end{equation}
\noindent Note that this value is also small when the gradient lies
primarily along the eigenvalues of smallest magnitude.
On each internal iteration, MR-QLP checks whether
either of these values is below a tolerance level
(the hyperparameter \texttt{rtol}; in our experiments, 5e-4)
and if either is, it ceases iteration.
With exact arithmetic,
either one or the other of these values
should go to 0 within a finite number of iterations;
with inexact arithmetic, they should just become small.
See~\cite{choi2011} for details.

Under this relaxed definition of gradient-flatness,
there will be a neighborhood of approximate gradient-flat points
around a strict, exact gradient-flat point
for functions with Lipschitz gradients and Hessians.
Furthermore, there might be connected sets of non-null Lebesgue measure
which all satisfy the approximate gradient-flatness condition
but none of which satisfy the exact gradient-flatness condition.
We call both of these \emph{gradient-flat regions}.

There are multiple reasonable numerical indices of flatness besides
the definition above.
For example,
the Hessian-gradient regularity condition in~\cite{roosta2018},
which is used to prove convergence of Newton-MR,
would suggest creating a basis for the approximate kernel of the Hessian
and projecting the gradient onto it.
Alternatively, one could compute the Rayleigh quotient
of the gradient with respect to the Hessian.
Our method has the advantage of being computed as part of the Newton-MR algorithm.
It furthermore avoids diagonalizing the Hessian or the specification
of an arbitrary eigenvalue cutoff.
The Rayleigh quotient can be computed with only one Hessian-vector product,
plus several vector-vector products,
so it might be a superior choice for larger problems where
computing a high-quality inexact Newton step is computationally infeasible.

\section{Gradient-Flat Regions Abound on Several Neural Network Losses}%
\sectionlabel{gfr-results}

\section{Conclusion}%
\sectionlabel{conclusion}

\onlyinsubfile{\printbibliography}

\end{document}
